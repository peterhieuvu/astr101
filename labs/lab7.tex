\documentclass[12pt]{article}
    \usepackage[utf8]{inputenc}
    \usepackage[margin=1in]{geometry}
    \usepackage[english]{babel}
    \usepackage{amsmath}
    \usepackage{fancyhdr}
    \usepackage{mathabx}

\title{\vspace{-1.5cm}Lab 7}
\author{Peter Hieu Vu\\ASTR\&101}
\date{June 5, 2018}

\pagestyle{fancy}
\lhead{Lab 7}
\rhead{Vu}
\renewcommand{\headrulewidth}{0pt}

\newcommand\lnend{\medskip\\}

\begin{document}
    \maketitle

    %\renewcommand{\labelenumi}{\Roman{enumi}}
    \begin{enumerate}
        \item Explain how you identified the Sun on the plot?\lnend{}
            I identified the Sun using the luminosity of $1L_{\Sun}$ and temperature of 5800K. The 5800K was simply found on the graph, and then the luminosity was plugged into the formula to get $\log (1) = 0$ since $10^0 = 1$.
        \item What is special about the stars on the main sequence?\lnend{}
            Stars on the main sequence are special because they all turn hydrogen into helium in their core.
        \item Describe in detail why these are called Red Giants. What is the implication of the "Red" in the name? Explain the "Giant" in the name. How is it that we can infer that these stars have large radii?\lnend{}
            The stars in the top right are Red Giants. They are red because of their temperature around 5000K. Despite low temperatures, they are very luminous which means they must be very large. This is according to Stefan's Law which says:\\Luminosity $\propto$ Area $\times$ $\mathrm{T}^4$.
        \item Describe in detail why these stars are called White Dwarfs. What is the implication of "White" in the name? Explain the "Dwarf" in the name. How is it that we can infer that these stars have small radii?\lnend{}
            White dwarfs are under the main sequence in the bottom right. They are white because their surface temperature gives them a whitish color. They are relatively hot compared to the sun, but less luminous which means they must be small in size. This is again according to Stefan's Law.
        \item What portion of the H-R diagram do most of the bright stars you plotted fall on? What does this tell you about the intrinsic luminosity of these stars?\lnend{}
            Most of the bright stars are in the top half of the H-R diagram. This means that they are intrinsically more luminous than the Sun.
        \item What portion of the H-R diagram do most of the nearby stars you plotted fall on? What does this tell you about the intrinsic luminosity of these stars?\lnend{}
            Most of the nearby stars lie in the bottom half of the H-R diagram. This means that they are intrinsically less luminous than the Sun.
        \item Describe one way in which astronomers make use of the H-R diagram other than to classify stars into different types.\lnend{}
            Astronomers can also use the H-R diagram to estimate certain characteristics about a star. One of these characteristics is mass which can be estimated based on where a star is on the main sequence, with stars on the left side more massive than stars on the right side. 
    \end{enumerate}
\end{document}