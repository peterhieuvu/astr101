\documentclass[../hw2.tex]{subfiles}

\begin{document}
    \lhead{CH 12}
    \sftitle{Homework Chapter 12 (After Midterm)}{June 12, 2018}
    
    \sfsection{Multiple Choice and True/False} %16, 18, 19, 22
    \begin{enumerate}
        \inum{16}\item \ans{b} During the vertical portion of the process to become a low mass star, the luminosity falls, but temperature remains nearly constant. This is because luminosity is on the vertical axis while temperature is on the horizontal axis.
        \inum{18}\item \ans{a} When the Sun becomes a red giant, its density will decrease and its luminosity will increase. With Stefan's law, we know that luminosity is proportional to area. With the Sun expanding to a red giant, the luminosity will increase, but since the mass stays roughly the same, the density will decrease.
        \item \ans{b} As a low mass star dies, it moves across the top of the H-R diagram because we see deeper into the star. The dying star's outer envelop begins to expand and gets less and less dense which eventually begins to reveal the hot core.
        \inum{22}\item \ans{a} If a low mass star is in a close binary system, it still can become a nova or supernova if it can pull material from the sister star. 
    \end{enumerate}
    
    \sfsection{Conceptual Questions} %28, 31, 33, 36, 38

    \begin{enumerate}
        \inum{28}\item Stars with more mass will have more gravity and thus create nuclear reactions at a much faster rate. This means that, while they have more material to use as fuel, they run through their hydrogen faster than less massive stars.
        \inum{31}\item Stars in binary systems can have a different destiny than what might be initially determined by its mass. In a binary system, mass can be transferred between the stars. For that reason, a star destined originally to become a white dwarf can become a nova or supernova.
        \inum{33}\item When a star runs out of nuclear fuel in its core, the core begins to compress, increasing the gravity and the energy generation in the hydrogen shell around it. The star, therefore releases more energy, but also expands which causes it too be cooler at the surface.
        \inum{36}\item If you were an astronomer making a survey of the observable stars in our galaxy, the chances of seeing a star undergoing the helium flash would be very slim. The helium flash takes place in only several hours in contrast to the billions of years of the lifetime of a star. There are many stars in the sky, but we don't see every single one yet and so the chances are even more slim.
        \inum{38}\item As an AGB star evolves into a white dwarf, it runs out of nuclear fuel, but is left with the hot core. Cores of stars are much hotter than surface temperatures, and so as a star becomes a white dwarf and the outer envelope begins to expand, more and more of the hot core is revealed until only a white dwarf is left.
    \end{enumerate}
\end{document}