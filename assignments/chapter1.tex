\documentclass[../hw1.tex]{subfiles}

\begin{document}
    \sfsection{Summary and Self Test}
    \begin{enumerate}
        \item In increasing size, the following goes: Earth, Sun, Solar System, Milky Way Galaxy, Local Group, Virgo Supercluster, Universe.
        \item \ans{c} If we compare our place in the universe with a very distant place, all of the laws of physics are the same in each place. This is important because, otherwise, our observations and ideas about the universe would be much less useful if they only explained what happened on Earth.
        \inum{6}\item In order of size, the following goes: radius of the Earth, a light minute, the distance from the Earth to the Sun, a light hour, the radius of the Solar System, and then a light year.
        \inum{8}\item\ans{b, d, a, c, e}\begin{enumerate}
                \inum{2}\item Hydrogen and helium are made in the big bang. This must happen first so that the Sun can be made.
                \inum{4}\item Stars are born and process light elements into heavier ones. These stars form from the elements in the previous step and allow for the creation of heavier elements which are involved in the following steps.
                \inum{1}\item Stars die and distribute heavy elements into the space between the stars. The exploding stars move the elements around space.
                \inum{3}\item Enriched dust and gas gather into clouds in interstellar space. The stuff that comes out of the previous step collects together for the next step to happen.
                \inum{5}\item The Sun and planets form from a cloud of interstellar dust and gas. Finally the components from the previous steps lead to the creation of the Sun and planets.
            \end{enumerate}
    \end{enumerate}
    %Questions and Problems
    \sfsection{True/False and Multiple Choice }
    \begin{enumerate}
        \inum{19}\item\ans{d} The Big Bang created hydrogen, helium and lithium, but not carbon.
        \inum{21}\item\ans{c} Occam's razor states that if two hypotheses fit the facts equally well, choose the simpler one.
        \item\ans{a} The cosmological principle states that on a large scale, the universe is the same anywhere at a given time.
    \end{enumerate}
    
    \sfsection{Conceptual Questions and Problems}
    \begin{enumerate}
        \inum{30}\item When a star explodes in the Andromeda Galaxy, it takes 2.5 million years for us to see it on Earth since it is 2.5 million light years away.
        \item When scientists say we are made of stardust, they mean that we are made of elements that form stars and come from the explosion of stars. The Big Bang created only hydrogen, helium, and small amounts of lithium. Heavier elements that we are made of come from in the centers of stars or in supernova explosions.
        \item Falsifiable means that something can be proved false by experimental or observational evidence. Saying that there is an afterlife isn't falsifiable because theres no way to know whether there is or isn't one unless you die. Saying "you can live without food or water," however, \emph{is} falsifiable because you can do experiments to see if the hypothesis is false.
        \item In common language, "theory" is used as another word for guess. Scientists, however, use "theory" to explain something based on observations and data that has been extensively tested, but never proven wrong (or at least yet). 
        \item A hypothesis is simply an idea that can be tested and which a prediction can be made on. A theory, however, has been tested extensively with regard to many factors which can be used to explain something. 
        \item If a discrepancy between scientific fields were found, the two ideas may be tested and modified until the conflict is resolved. 
        \inum{37}\item The fact that textbooks change shows that our knowledge can morph and change as we learn more. Over time, technology advances, ideas pop up, and this all can create opportunities to improve our understanding of the universe. Our scientific facts then end up evolving as we learn more.
    \end{enumerate}
    
    \sfsection{Problems}
    \begin{enumerate}
        \inum{43}\item For this problem, we can assume that the car goes 60mph on average during a trip from New York to Los Angeles. We know that that $d = vt$ with $d$ representing distance, $v$ representing velocity and $t$ representing time. If we solve for time, we get $t = \frac{d}{v}$. Plugging in the numbers, we are left with $t = \frac{2444\mathrm{miles}}{60\mathrm{mph}} = 40.733$ car-hours. Since there are 24 hours in a day, this would be $\frac{40.733}{24} = 1.697$ car-days. For walking, we can assume that walking speed is 3 miles per hour on average. This would mean it would be 20 times longer equaling 814.67 walking-hours or 33.94 foot-months. There are 365 days in a year and 12 months in a yar so this would be 1.116 foot-months or 0.092998 foot-years.
    \end{enumerate}
\end{document}