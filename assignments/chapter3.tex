\documentclass[../hw1.tex]{subfiles}

\begin{document}
    \sfsection{Summary and Self Test}
    \begin{enumerate}
        \item \begin{enumerate}
            \item A planet moves fastest when it is \underline{closest} to the sun and slowest when it is \underline{furthest} from the Sun.
            \item Each ellipse has two foci. At one focus is the \underline{Sun}. At the other focus is \underline{nothing}.
            \item A planet with a period of 84 Earth years has an orbit that is \underline{larger} than a planet with an orbit of 1 Earth year.
        \end{enumerate}
        \inum{6}\item Eccentricity is how far the foci are apart which corresponds to how flat the ellipse it. The eccentricity of a circular orbit is zero. This would be when both foci are on top of each other.
        \inum{9}\item\ans{a, b, c, d} A net force must be acting when an object accelerates, an object changes direction but not speed, an object changes speed but not direction, or an object changes speed and direction. This is because net force is required to create any acceleration and acceleration is any change in velocity, meaning any change in direction or speed, including combinations of both.
        \item\ans{a} If I was transported to a planet with twice the mass of Earth, but the same radius of Earth, my weight would be increased by a factor of 2. Newton’s universal law of gravitation says: \[F_\mathrm{gravity} = \frac{G m_1 m_2}{r^2}\]The thing that changes is the mass of the planet and multiplying that by two will result in two times the force due to gravity.
    \end{enumerate}

    \sfsection{Multiple Choice and True/False}
    \begin{enumerate}
        \inum{23}\item\ans{c} Both forces are the same according to Newton’s Third Law. The formula for gravitational force is $\frac{G m_1 m_2}{r^2}$ and this is applied on both objects equally.
        \inum{25}\item Since the average distance from Uranus from the Sun is 19 times Earth’s distance from the Sun. The Sun’s gravitational force on Uranus is 361 times weaker than the sun’s gravitational force on Earth. From the formula in the answer to question 23, we can see that the denominator contains an $r^2$ which means 19 times greater distance will mean $19^2=361$ times weaker gravitational force.
    \end{enumerate}

    \sfsection{Conceptual Questions}
    \begin{enumerate}
        \inum{26}\item The semi-major axis is especially important because the length of it is equal to the average distance of a planet’s orbit to the object it is orbiting. Additionally, according to Kepler’s third law, this average distance (d) is related to the orbital period (p) in that $p^2\propto d^2$.
        \item Inertia is an object’s resistance to a change in motion. The more inertia an object has, the harder it is to accelerate that object. This is related to mass because an object with more mass will have more inertia than an object with less mass.
        \inum{35}\item Weight and mass are different because weight is a measurement of force due to gravity while mass is a measurement of the amount of matter inside an object. The mass of an object remains the same regardless of what object it is on or how much gravity it is being subject to. This is different than weight which depends on the object it is on and what forces are acting upon it.
        \inum{37}\item A bound orbit is where an object is gravitationally bound to the object it is orbiting. This means that the orbiting object will stay in orbit and will orbit in the shape of an ellipse. An unbound orbit is when an object begins to orbit another object, but reaches a velocity faster than the escape velocity and continue to leave orbit.
        \inum{39}\item If astronomers discovered an object approaching the Sun in an unbound orbit, that would mean that the object came from outside the Solar System. If the object was a part of the Solar System, it would be in a bound orbit around the sun. As such, an object approaching in an unbound orbit will have come from outside the Solar System.
    \end{enumerate}

    \sfsection{Numerical Problems}
    \begin{enumerate}
        \inum{41}\item We can solve the for the period based on distance using Kepler’s Third Law.
            \begin{align*}
            d^3 &\propto p^2\\
            d_{earth}^3 &\propto p_{earth}^2 \\
            d_{obj}^3 &\propto p_{obj}^2 \\
            \left(\frac{d_{obj}}{d_{earth}}\right)^3 &= \left(\frac{p_{obj}}{p_{earth}}\right)^2 \tag{then plug in values}\\
            \left(\frac{46.4\;\mathrm{AU}}{1\;\mathrm{AU}}\right)^3 &= \left(\frac{x\;\mathrm{Earth\,Years}}{1\;\mathrm{Earth\,Year}}\right)^2\\
            x = 46.5^{3/2} &\approx \boxed{316.1 \;\mathrm{Earth\,Years}}
        \end{align*}
        \inum{49}\item \begin{enumerate}
            \item Just as you fall out of the airplane your gravitational acceleration would be $9.8m/s^2$ as that is the approximately the acceleration due to gravity on the surface of Earth. Depending on how high in the air, it could be a bit less.
            \item This acceleration would be the same as if you were strapped to a flight instructor and had twice the mass. The force due to gravity depends on the mass of the object subject to that gravity, however, the acceleration due to gravity stays constant regardless of mass.
            \item Just as you fall out of the airplane, the gravitational force on you, assuming your mass is 70kg, would be around 686N. $F=ma$ and so $70kg*9.8m/s^2=\boxed{686N}$
            \item The gravitational force would be bigger if you were strapped to a flight instructor and so had twice the mass. Looking at the formula, doubling the mass would double the force as acceleration stays constant. $(2m)*a=2F$. Additionally, the universal law of gravitation says the same thing. $\frac{G(2m_1)m_2}{r^2}=2F$.
        \end{enumerate}
    \end{enumerate}
\end{document}