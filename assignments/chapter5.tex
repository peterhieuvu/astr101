\documentclass[../hw2.tex]{subfiles}

\begin{document}
    \lhead{CH 5}
    \sftitle{Homework Chapter 5 (After Midterm)}{May 21, 2018}

    \sfsection{Summary and Self-Test}
    \begin{enumerate}
        \item \ans{b,e,f} Gravity does not determine the direction in which the system rotates since it is the same in all directions. The pull of it causes the cloud to collapse, the smaller particles coming together to form larger particles. Once bodies are large enough, gravity pulls them together to make even larger bodies. Gravity is required for atmospheres to form around planetesimals and to hold them on the object.
        \item \ans{c} When dust dust grains first begin to grow into larger objects, this occurs because of collisions between dust grains. These dust grains don't have enough gravity to pull each other in, but they \emph{can} combine by colliding into each other.
        \item \ans{c} The direction of revolution in the plane of the Solar System was determined by the rotation of the original cloud. The net angular momentum of the cloud was counter-clockwise and so since the Solar System formed from that cloud, it also has counter-clockwise angular momentum and spins that way.
        \item \ans{b} The terrestrial planets are different from the giant planets because when they formed, the inner Solar System was hotter than the outer Solar System. In the hotter inner ring, volatile materials cannot form and so the planets there are made primarily of refractory materials which are more dense and smaller. 
        \inum{7}\item \ans{a} Nuclear reactions require very high temperature and density. The atoms need to be pressed together with high temperatures like what happens inside the core of a star for nuclear reactions to happen.
        \inum{10}\item \ans{d} A planet in the "habitable zone" is at a distance where liquid water can exist on the surface. Life on Earth needs water to exist and so we define "habitable zone" to be a region where liquid water can exist on the surface of a planet.
    \end{enumerate}
    
    \sfsection{Multiple Choice and True/False}
    \begin{enumerate}
        \inum{19}\item \ans{a} Molecular clouds collapse because of gravity. Gravity pulls the particles of dust together and collapse towards the center of mass of the cloud.
        \item \ans{a} Because angular momentum is conserved, an ice-skater who throws her arms out will rotate more slowly. By throwing her arms out, she is increasing her moment of inertia and since angular momentum is conserved, rotational speed must reduce.
        \inum{23}\item \ans{d} Jupiter still has its primary atmosphere. Primary atmospheres are made of hydrogen and helium which don't last long unless the planet has enough gravity to hold the atmosphere. Jupiter is massive enough to keep its primary atmosphere.
        \item \ans{e} Extrasolar planets have been detected by the spectroscopic radial velocity method, the transit method, gravitational lensing, and direct imaging. %Check this!!!!!!!!!!!
    \end{enumerate}
    
    \sfsection{Conceptual Questions}
    \begin{enumerate}
        \inum{30}\item The law of conservation of angular momentum controls a figure-skater's rate of spin based on that figure-skater's moment of inertia. Since angular momentum is conserved, a figure-skater can change their rate of spin by using their body to increase or decrease their moment of inertia. Moving their arms out allows the skater to spin slower, and moving their arms in allows the skater to spin faster.
        \inum{33}\item The inner part of an accretion disk is hotter than the outer part since it is closer to the forming protostar which radiates heat to the material nearby. Another reason is because the particles closer to the center experience more gravity and are pulled longer distances, having more opportunity to collide with other particles and gain thermal energy.
        \inum{35}\item The original atmospheres of the terrestrial planets fly away into space. Original atmospheres are made up of hydrogen and helium which are light and require strong gravity to keep together. Larger planets can hold their primary atmosphere, but the smaller and less massive terrestrial planets do not have strong enough gravity to keep their original atmospheres intact.
        \item After the last of the planets formed, the leftover Solar System debris have settled mostly in their own places. Some small pieces, meteoroids, are floating about through the Solar System. Other debris has have been sent to places like the asteroid belt, the Kuiper belt, and the Oort cloud.
        \inum{39}\item Originally, astronomers have understood that larger planets of volatile materials generally form further away from their parent star. After finding large planets in close orbits with their parent star, they have found that those planets may have formed further away and then spiraled in to end up much closer to their parent star.
    \end{enumerate}
\end{document}