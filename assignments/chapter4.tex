\documentclass[../hw1.tex]{subfiles}

\begin{document}
    \sfsection{Summary and Self Test}
    \begin{enumerate}
        \inum{5}\item\ans{c} Light acts as both a wave and a particle. In some experiments light is observed to act as a wave, but in others it seems like it acts as a particle.
        \inum{8}\item\ans{a, e} Radio waves and visible light can be observed from the ground. Higher energy waves like ultraviolet light, x-ray light, and gamma radiation do not pass well through the atmosphere to the ground.
        \item\ans{e, c, b, d, a} Radio waves have the longest wavelength, followed by infrared radiation, visible light, ultraviolet light, and finally gamma rays.
        \item\begin{enumerate}
            \item aperture is \ans{3} the diameter of the largest lens or mirror of the telescope
            \item resolution affects the \ans{4} ability to distinguish objects that appear close together in the sky
            \item focal length is \ans{2} the distance from lens to focal plane
            \item chromatic aberration is \ans{6} a rainbow-making effect
            \item diffraction is \ans{7} a smearing effect due to sharp edges
            \item an interferometer is \ans{1} several telescopes connected to act as one
            \item adaptive optics is \ans{5} computer-controlled active focusing which helps to correct for the distortion that occurs in Earth's atmosphere
        \end{enumerate}
    \end{enumerate}

    %Questions and Problems
    \sfsection{Multiple Choice and True/False}
    \begin{enumerate}
        \inum{17}\item\ans{d} Light at the lowest energy end of the electromagnetic spectrum is in the radio region. That region is the area with the longest wavelength and the lowest frequency which correlates with low energy.
        \inum{20}\item\ans{b} If the wavelength of a beam of light were halved, the frequency would be two times larger. The wave speed of a beam of light stays constant and since $c = f \lambda$, halving the wavelength will result in double the frequency.
        \inum{24}\item\ans{b} Gamma ray telescopes are placed in space because gamma rays do not penetrate the atmosphere. Gamma rays are capable of destroying atoms in the atmosphere so as they pass through Earth’s atmosphere, the energy is dissipated and the rays do not reach the ground.
    \end{enumerate}
    
    \sfsection{Conceptual Questions}
    \begin{enumerate}
        \inum{28}\item The energy of a single photon is different than the energy of a beam of light. If photons of blue light have more energy than photons of red light, a beam of red light can carry as much energy as a beam of blue light if the beam of red light has more photons. Since photons of blue light have more energy, less is required to create a beam of the same energy as that of a red beam, but it is still possible to have beams of equal energy given that the energy and photons are available.
        \inum{36}\item One way Earth’s atmosphere interferes with astronomical observations is by distorting visible light. As light passes through the atmosphere, the small bubbles of air act like weak lenses which, by the time it might reach the telescope, the image can become fuzzy and distorted. Earth’s atmosphere also interferes with astronomical observations by preventing certain wavelengths of electromagnetic radiation from passing through. Higher energy waves like gamma rays, x-rays, and ultraviolet rays do not pass well through the atmosphere and so it becomes more difficult to observe those kinds of radiation. Finally, light from the ground can also be reflected by the atmosphere, changing the brightness of the night sky.
        \inum{40}\item The idea that we put astronomical telescopes in orbit to get closer to the objects we are observing is incorrect. First off the reason we put telescopes in orbit is not to get closer to the objects but to be able to make observations without the atmosphere in the way. Still, even if that were not the case, putting the telescopes in orbit does not get us significantly closer to the objects we are observing. The scale of the universe is enormous and the distance the telescopes are away from Earth is insignificant to the distance between Earth and the objects that we are observing with those telescopes.
    \end{enumerate}
\end{document}