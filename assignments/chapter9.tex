\documentclass[../hw2.tex]{subfiles}

\begin{document}
    \lhead{CH 9}
    \sftitle{Homework Chapter 9 (After Midterm)}{May 28, 2018}

    \sfsection{Summary and Self-Test} %5,6,7,8,9
    \begin{enumerate}
        \inum{5}\item \ans{d} As a comet leaves the inner Solar System, its ion tail always points away from the Sun. The ion tail happens because of how the solar wind blows away material from the comet.As such, the ion tail always points away from the Sun, even when it it leaves the inner Solar System where the tail would be in front of the comet. 
        \item \ans{a,b,c,d} Meteorites can tell us about the composition of the Solar System as well as the composition of asteroids.  These meteorites are formed with the Solar System and because of that, they can tell us about the early composition of the Solar System. Additionally, some meteorites have come from parts of asteroids, meteors, or even other planets which means they could also tell us something about those objects.
        \item \ans{a,b,d,e} Titan resembles early Earth because it has an atmosphere of mostly nitrogen, terrain similar to Earth's, and is rich in organic compounds. On the surface, Titan has hills and ridges thought to be formed from its methane cycle which is similar to what water does on Earth. The organic compounds in the ground also are similar to what could have been on the early Earth. It also has a thick atmosphere like early Earth.
        \item \ans{c,e} Pluto differs significantly from the eight Solar System planets in that its orbit is chaotic and that it has not cleared its orbit. All of the eight classical planets have near circular orbits which are close to the ecliptic plane. Pluto, however, has an orbit off of the ecliptic plane and which is so eccentric it sometimes comes inside the orbit of Neptune. Pluto also lives inside of the Kuiper belt around many other objects.This is unlike the other planets who are essentially alone in their orbits.
        \item \ans{b} If an asteroid is not spherical, it tells you that its mass is low. An asteroid with high mass will have enough gravity to keep a more spherical shape while a low mass asteroid, when damaged, will not be able to keep a spherical shape.
    \end{enumerate}
    
    \sfsection{Multiple Choice and True/False} %19,23,25
    \begin{enumerate}
        \inum{19}\item \ans{a} Short and long period comets differ because short-period comets orbit prograde while long-period comets have either prograde or retrograde orbits. Short period comets exist in the Kuiper belt which revolve around the Sun usually in the same direction while comets outside in the Oort cloud orbit in all directions.
        \inum{23}\item \ans{a} A meteoroid is found in space, a meteor is found in the atmosphere, nd a meteorite is found on the ground.  
        \inum{25}\item \ans{b} During a meteor shower, all meteors trace back to a single region in the sky because all the meteors area traveling in the same direction relative to Earth. Meteor showers happen when Earth passes into the orbit of a comet and run into all that dust and debris. Because they all come from comet, the meteors are all moving in the same direction when they enter the atmosphere and will look like they are originating from one region in the sky.
    \end{enumerate}
    
    \sfsection{Conceptual Questions} %26, 27, 28, 29, 31, 32, 34 ,37, 38, 40
    \begin{enumerate}
        \inum{26}\item Astronomers are especially interested in the asteroids whose orbits cross that of the Earth because those asteroids could potentially collide with the Earth. An asteroid collision could cause massive damage to the Earth and life on Earth and so they are studied since they could directly affect us.
        \item Tidal heating drives volcanism on Jupiter's moon Io. As Io orbits Jupiter, Jupiter's gravity pulls and morphs Io as it orbits in its elliptical orbit which generates enough energy to melt part of it and create geological activity. 
        \item Cryovulcanism is essentially vulcanism, but with volatiles like water instead of molten rock. This means that the materials are liquid at very low temperatures. With cryovulcanism, solid forms of volatiles are melted and then spurted out in plumes.
        \item Europa is thought to possibly be geologically active because its icy slabs have seemed to either shift or split apart. Titan is thought to be geologically active because of how the methane in its atmosphere still existed when it should have been destroyed by solar radiation. Geological activity could have renewed the methane in its atmosphere.
        \inum{31}\item Titan contains abundant amounts of methane which requires an explanation because methane should have been destroyed by ultraviolet solar radiation. Photodissociation is the process that destroys methane in Titan's atmosphere.  
        \item Some moons, while not currently geologically active, have surface qualities that may provide evidence of past geological activity. Things like ripples on the surface can be evidence of the flow of magma on the surface and shallow craters with bulges can be evidence of deformations which have been partially filled in by geological activity.
        \inum{34}\item The Kuiper Belt is closer to the Sun than the Oort Cloud is and is also significantly smaller. Objects in the Oort Cloud also orbit and rotate in all directions and at different angles. This is significantly different than objects in the Kuiper Belt which orbit and rotate in the same direction as the inner Solar System, and also more closely along the ecliptic plane. Finally, Oort Cloud objects have very elongated orbits with periods up to millions of years while Kuiper Belt objects have less elongated orbits with periods of under 200 years.
        \inum{37}\item Meteorites, asteroids, and comets formed at the same time as the Solar System and, as such, can be used to study the Solar System. Earth is simply one object in the Solar System, there is a lot more out there. Studying other objects can help us understand more about the Solar System and how it formed.
        \item We should be concerned about the possibility of a collision because, while they are rare, they can still happen and do happen. If we collide with an asteroid, it could drastically affect the Earth and the life on it. An asteroid or comet collision could potentially be prevented and so it is worth being concerned about.
        \inum{40}\item A comet is significantly larger than a meteor and also can be seen for much longer. A comet becomes visible when it nears the Sun, and so can be visible for a while whereas a meteor will simply enter the Earth's atmosphere and likely disintegrate in short time. That also means the comet is usually much further away, being close to the sun rather than inside the Earth's atmosphere itself. 
    \end{enumerate}
\end{document}