\documentclass[../hw2.tex]{subfiles}

\begin{document}
    \lhead{CH 4 pt.2}
    \sftitle{Homework Chapter 4 (After Midterm)}{May 15, 2018}

    \sfsection{Summary and Self Test}
    \begin{enumerate}
        \inum{10}\item \begin{enumerate}
            \item aperture is \ans{3} the diameter of the largest lens or mirror of the telescope
            \item resolution affects the \ans{4} ability to distinguish objects that appear close together in the sky
            \item focal length is \ans{2} the distance from lens to focal plane
            \item chromatic aberration is \ans{6} a rainbow-making effect
            \item diffraction is \ans{7} a smearing effect due to sharp edges
            \item an interferometer is \ans{1} several telescopes connected to act as one
            \item adaptive optics is \ans{5} computer-controlled active focusing which helps to correct for the distortion that occurs in Earth's atmosphere
        \end{enumerate}
    \end{enumerate}

    \sfsection{Questions and Problems}
    \begin{enumerate}
        \inum{22}\item\ans{b} Astronomers put telescopes in space to avoid atmospheric effects. The atmosphere, for one, blocks most of electromagnetic waves other than radio and visible wavelengths. That means those waves simply don't reach the ground. Additionally, air bubbles in the atmosphere can distort images on the way down to the ground. Putting telescopes in space gets around those issues.
        \inum{29}\item The primary disadvantage of using a simple lense in a refracting telescope is chromatic aberration. This can mess with the image, potentially creating fuzzing around the edges of a variety of colors. A compound lens is made up of two different materials which helps to correct for the chromatic aberration.
        \inum{31}\item One advantage reflecting telescopes have over refractors is how many polished surfaces need to be made to create a "perfect" image. In a refracting telescope, there must be 3 lenses with two surfaces each, totaling 6 polished surfaces in order to make a telescope without chromatic aberration. With a reflecting telescope, the light never enters the glass and so at minimum there only needs to be one polished surface to create a perfect image. Another advantage is weight and weight distribution. A refractor has the three glass lenses versus the single piece of glass in a reflecting telescope. Additionally, the weight in a reflecting telescope is at the bottom (allowing for a more stable design) versus the weight in a refractor being at the top.
        \inum{34}\item Manufacturers of quality refracting telescopes and cameras correct for the problem of chromatic aberration by using compound lenses. These compound lenses are made of two different types of glass which helps to correct for chromatic aberration. For highest end refracting telescopes, a type of compound lens called a triplet lense is used which takes advantage of 3 single lenses to create a perfect image without any chromatic aberration.
    \end{enumerate}

    \sfsection{Numerical Problems}
    \begin{enumerate}
        \inum{43}\item \begin{align*}
            \text{light-gathering ability} &\propto \text{area}\\
            \text{diameter}^2 &\propto \text{area}\\
            \frac{16}{4} &= 4\tag*{so}\\
            4 * \text{diameter} &\implies 4^2=16 * \text{area}\\
            16 \text{area} &\implies 16 * \text{light-gathering ability}
        \end{align*}
        Therefore by increasing the diameter from 4 to 16, the light-gathering power increases by a factor of $4^2$ or 16.
    \end{enumerate}
\end{document}