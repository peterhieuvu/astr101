\documentclass[../hw2.tex]{subfiles}

\begin{document}
    \lhead{CH 10}
    \sftitle{Homework Chapter 10 (After Midterm)}{May 29, 2018}

    \sfsection{Summary and Self-Test} %1, 2, 3, 4, 7
    \begin{enumerate}
        \item \ans{b} Star B will appear brighter in the night sky because it apparent brightness is inversely proportional to distance squared.
        \item \ans{a} Star A is hotter because peak wavelength is inversely proportional to temperature. Since star A is blue, it has a shorter wavelength than star B and should be hotter.
        \item \ans{b} Star B is more massive. Luminosity over temperature to the fourth power is proportional to radius squared. The luminosity of the stars are equal, so since star A is hotter, it should be smaller than star B. 
        \item \ans{a,b} If a star has very strong hydrogen absorption lines, the temperature is right for hydrogen to make lots of transitions, and hydrogen is abundant in the star because hydrogen is absorbing at these wavelengths. With strong absorption lines, the temperature must be just right to make many transitions to different energy states. Also, the fact that it matches up with hydrogen absorption lines means there is a lot of hydrogen in the star to absorb those specific wavelengths.
        \inum{7}\item \ans{a} If a star has very weak hydrogen lines and is blue, it most likely means the star is too hot for hydrogen lines to form. A blue star is very hot and isn't the optimum temperature for hydrogen lines to form. It is unlikely for the star to have no hydrogen since hydrogen is the simplest atom. 
    \end{enumerate}
    
    \sfsection{Multiple Choice and True/False} %16, 24, 25
    \begin{enumerate}
        \inum{16}\item \ans{b} If two stars have equal luminosities, but star A has a much larger radius, we can say that star A is cooler than star B. Luminosity per unit area is proportional to temperature to the fourth power. If star A has a larger radius, it would have less luminosity per unit area and thus be cooler than star B.
        \inum{24}\item \ans{a} If a star is found directly above the Sun on the H-R diagram, we can conclude that it is more luminous than the Sun. Luminosity is on the up and down axis with increasing luminosity going up.
        \item \ans{d} If a star has the same temperature as the Sun, we can not say anything about its mass without knowing if it was on the main sequence or not.
    \end{enumerate}
    
    \sfsection{Conceptual Questions} %28, 29, 33, 35, 37, 39
    \begin{enumerate}
        \inum{28}\item If we were on the planet Mars, we would be able to use stellar parallax better. Since Mars has a larger orbit, there is a greater maximum distance possible and so we can measure the parallax of stars that are farther away than what we can do on Earth. On Jupiter, this is even further exaggerated with an even larger orbit than Mars or Earth. On Venus the effect would be the opposite and we would have more trouble measuring stellar parallax. 
        \item If viewers describe the brighter star as golden and the fainter one as sapphire blue, we can guess that the sapphire blue star is hotter since blue light has a longer wavelength than yellow. Additionally, since the brighter one is the golden one, the golden one must be significantly larger than the sapphire looking star. Luminosity over radius squared is proportional to temperature to the power of four and so the golden one must be large enough to keep the proportionality true.
        \inum{33}\item In this spectrum, it is a white light source because all colors are represented in the spectrum. The spectrum tells us that there is a cool cloud of gas because there are gaps in the spectrum for wavelengths with the correct energy to be absorbed by the gas. If the cool cloud of gas were located behind the white light source, the spectrum would be full and even without breaks since the white light would not have to pass through the gas cloud.
        \inum{35}\item The stellar spectral types are not in alphabetical order because they have been rearranged. Originally, a spectral type A star had strongest spectral lines, and a spectral type B star had slightly weaker spectral lines. Eventually, some were removed and then they were reorganized based on surface temperature.
        \inum{37}\item The only stars whose mass we can measure directly other the Sun are stars in binary systems because Kepler's third law only applies in systems in binary systems where we can measure the center of mass and average distance between the stars. 
        \inum{39}\item We can estimate the mass of stars not in binary systems if they are on the main sequence based on where they are located on the main sequence.
    \end{enumerate}
\end{document}