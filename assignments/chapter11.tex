\documentclass[../hw2.tex]{subfiles}

\begin{document}
    \lhead{CH 11}
    \sftitle{Homework Chapter 11 (After Midterm)}{June 6, 2018}

    \newcommand\usc[1]{${}^#1$}

    \sfsection{Summary and Self-Test} %2,3,4,7,9
    \begin{enumerate}
        \inum{2}\item \ans{e=g=a=f,h=b,d=c} First, two hydrogen nuclei collide to become \usc{2}H. When this happens, a position, neutrino, and two gamma rays are emitted. In the next step, one deuterium nucleus and one hydrogen nucleus collide to get \usc{3}He, emitting one gamma ray. In the third step, two \usc{3}He collide and become \usc{4}He, releasing two hydrogen nuclei.
        \item \ans{c} As energy moves out from the Sun's core toward its surface, it first travels by radiation, then by convection, and then by radiation. Moving out from the core, the first zone is the radiative zone where there are no atoms to carry energy. Once the energy reaches the convection zone, there are atoms to carry the energy and the energy is carried by convection to the surface. Once reaching the surface, the energy is radiated into space.
        \item \ans{a} The physical model of the Sun's interior has been confirmed by observations of neutrinos and seismic vibrations. %LOOK INTO THIS MORE!!!
        \inum{7}\item \ans{c} Radiation transports energy by moving light. Photons carry the energy in electromagnetic waves without requiring a medium to move through. Convection transports energy by moving matter. Fluids are heated in a way which create a convection cycle, transferring energy between two things. 
        \inum{9}\item \ans{c} The solar wind creates a teardrop-shaped bubble around the Solar System. This is because the Sun is also moving and so the solar wind which is come from the Sun is dragged behind, forming a teardrop shape.%LOOK INTO THIS MORE!!!%
    \end{enumerate}
    
    \sfsection{Multiple Choice and True/False} %16, 18, 22, 23
    \begin{enumerate}
        \inum{16}\item \ans{b} Hydrostatic equilibrium inside the Sun means that radiation pressure balances the weight of outer layers pushing down.
        \inum{18}\item \ans{b} The proton-proton chain doesn't happen spontaneously on Earth because the process requires very high temperatures and pressures. For nuclear fusion to happen, there needs to be enough temperature and pressure to get the positively charged hydrogen protons to get close enough to stick together by the strong nuclear force.
        \inum{22}\item \ans{d} If the Sun were to suddenly burn an abnormally large number of hydrogen in its core, the first thing that would observe is the emission of more neutrinos. It would take a long time for the light inside the core to reach the outer layers and so we would not see it become brighter or bluer at first. Similarly, it would also take time for the increased pressure in the core to affect the Sun visibly because of inertia. Neutrinos, however, move very fast and do not interact with matter which means that it can go straight out into space.
        \item \ans{d} The corona isn't much much brighter than the photosphere despite significantly higher temperatures because the corona has much lower density. %LOOK INTO THIS MORE 
    \end{enumerate}
    
    \sfsection{Conceptual Questions} %26, 30, 33, 35, 39

    \begin{enumerate}
        \inum{26}\item Hydrostatic equilibrium is the balance of pressures. With all of the nuclear reactions happening in the core, there is pressure outwards towards the surface. With the large mass of the Sun, however, there is the force of gravity counteracting the radiation pressure. This creates an equilibrium. There is a similar balance with energies and the Sun. The energy that is generated in the Sun's core is equal to the energy that is emitted from the Sun. This keeps the energy of the Sun the same so that it maintains its shape. 
        \inum{30}\item Nuclear fusion is the fusion of light atoms to form a heavier one. One example of this is the proton-proton chain which creates a Helium nucleus from 4 hydrogen nuclei. A helium nucleus is slightly less massive than 4 hydrogen nuclei, however, so the remaining mass is converted to energy. This relates to the Sun's source of energy because the proton-proton chain is what happens inside of the Sun's core.
        \inum{33}\item To start, the proton-proton chain requires 4 hydrogen nuclei. Two combine, creating a deuterium nucleus, as well as a positron, neutrino, and two gamma rays. The remaining two hydrogen nuclei combine in the same way. Each of the deuterium nuclei then combine with another hydrogen nucleus to make a helion each along with a gamma ray each. Finally, these helions combine, creating a helium nucleus, emitting two leftover hydrogen nuclei.
        \inum{35}\item In the proton-proton chain, even though six hydrogen nuclei are involved, we only include four because two are emitted at the end. In essence, there are two recycled hydrogen nuclei which means that only four are "consumed" or used in the process.
        \inum{39}\item To make models of the Earth's interior, we make predictions and measurements of how seismic waves travel through the Earth. Scientists then make measurements of the Sun's vibrations to find out about the Sun's interior with helioseismology. 
    \end{enumerate}
\end{document}